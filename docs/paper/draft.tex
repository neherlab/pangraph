\documentclass[aps,rmp,superscriptaddress,notitlepage,10pt]{revtex4-1}
\usepackage{amsmath,amsthm,amsfonts,amssymb,amscd}
\usepackage{hyperref}
\usepackage{graphicx}
\usepackage{wrapfig}
\usepackage{enumerate}
\usepackage[utf8x]{inputenc}

\begin{document}
\title{PanGraph: scalable multiple genome alignment for pan-genome analysis}
\author{Nicholas Noll}
\affiliation{Kavli Institute for Theoretical Physics, University of California, Santa Barbara}
\author{Marco Molari}
\affiliation{Swiss Institute of Bioinformatics, Basel, Switzerland}
\affiliation{Biozentrum, University of Basel, Basel, Switzerland}
\author{Richard Neher}
\affiliation{Swiss Institute of Bioinformatics, Basel, Switzerland}
\affiliation{Biozentrum, University of Basel, Basel, Switzerland}

\begin{abstract}
    Reference genomes, a simple coordinate system used to parameterize population alleles relative to a given isolate, fail to capture genomic diversity.
    As such, progress has focused on the elucidation of the \emph{pangenome}: the set of all genes observed within \emph{all} isolates of a given species.
    With the wide-spread usage of long-read sequencing, the number of high-quality, complete genome assemblies has increased dramatically.
    However, traditional computational approaches towards whole-genome analysis either scale poorly, or treat genomes as dissociated bags of genes, and thus are not suited for this new era.
    Here, we present PanGraph, a Julia based library and command line interface for aligning whole genomes into a graph, wherein each genome is represented as an undirected path along vertices, which in turn, encapsulate homologous multiple sequence alignments.
    The resultant data structure succinctly summarizes population-level nucleotide and structural polymorphisms and can be exported into a myriad of formats for either downstream analysis or immediate visualization.
\end{abstract}

\maketitle

\section{Introduction}
\section{Algorithms and implementation}
\section{Validation and performance}
\section{Discussion}

\end{document}
